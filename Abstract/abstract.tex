% !TEX TS-program = pdflatex
% !TEX encoding = UTF-8 Unicode

% This is a simple template for a LaTeX document using the "article" class.
% See "book", "report", "letter" for other types of document.

\documentclass[12pt]{article}
\usepackage[round, sort , authoryear]{natbib}
\usepackage[T1]{fontenc}
%%% PAGE DIMENSIONS
\usepackage[margin=2.5 cm]{geometry}
\usepackage{blindtext} % to change the page dimensions
\geometry{a4paper} % or letterpaper (US) or a5paper or....
% \geometry{margin=2in} % for example, change the margins to 2 inches all round
% \geometry{landscape} % set up the page for landscape
%   read geometry.pdf for detailed page layout information
\usepackage{tgpagella}
\usepackage[utf8]{inputenc} 
\usepackage{graphicx} % support the \includegraphics command and options
\usepackage{epstopdf}
\usepackage[hang]{footmisc}
\usepackage{lipsum}
\usepackage{setspace}
% \usepackage[parfill]{parskip} % Activate to begin paragraphs with an empty line rather than an indent
\usepackage{pgfplots}
\pgfplotsset{compat=1.13}
\usepackage{caption,fixltx2e}
\usepackage[flushleft]{threeparttable}
\usepackage{color, colortbl}
\definecolor{Gray}{gray}{0.9}
%%% PACKAGES
\usepackage{placeins}
\usepackage{booktabs} % for much better looking tables
\usepackage{array} % for better arrays (eg matrices) in maths
\usepackage{paralist} % very flexible & customisable lists (eg. enumerate/itemize, etc.)
\usepackage{verbatim} % adds environment for commenting out blocks of text & for better verbatim
\usepackage{subfig} % make it possible to include more than one captioned figure/table in a single float
% These packages are all incorporated in the memoir class to one degree or another...
\usepackage{amsmath}
\usepackage{cases}
\usepackage{graphicx}
\usepackage{float}
\usepackage{authblk}
\usepackage{pgfplots}
\usepackage{pdfpages}
\linespread{1.5}
\setlength{\footnotemargin}{4mm}
\usepackage{amssymb} 
\usepackage{tabularx}
\usepackage{subfig}
\addtolength{\footnotesep}{2mm} % change to 1mm
\usepackage{kpfonts}    % for nice fonts
\usepackage{microtype} 
\usepackage{booktabs}   % for nice tables
\usepackage[colorlinks=false, linktocpage=true]{hyperref}
%\usepackage[flushmargin]{footmisc}
%\addtolength{\footnotesep}{3mm} % change to 1mm
\hypersetup{
	colorlinks,
	linkcolor={red!50!black},
	citecolor={blue!50!black},
	urlcolor={blue!80!black}
}
% use for hypertext
\usepackage[colorinlistoftodos]{todonotes}
\newenvironment{customlegend}[1][]{%
	\begingroup
	% inits/clears the lists (which might be populated from previous
	% axes):
	\pgfplots@init@cleared@structures
	\pgfplotsset{#1}%
}{%
% draws the legend:
\pgfplots@createlegend
\endgroup
}%
%For Figures, below

\usepackage{tikz}
\usetikzlibrary{shapes}
\usepgflibrary{arrows} % LATEX and plain TEX and pure pgf
\usepgflibrary[arrows] % ConTEXt and pure pgf
\usetikzlibrary{arrows} % LATEX and plain TEX when using Tik Z
\usetikzlibrary[arrows] % ConTEXt when using Tik Z
\usepackage{hyperref}


%%% The "real" document content comes below...
\title{Cohabitation \textit{vs} Marriage:\\ Mating Strategies by Education in the USA}
\author{Fabio Blasutto\thanks{IRES, Université catholique de Louvain. E-mail: \tt{fabio.blasutto@uclouvain.be}}}

 \begin{document}
 %Stuff for figures, below

 	\tikzstyle{block} = [draw, fill=white, rectangle, 
 	minimum height=3em, minimum width=6em]
 	\tikzstyle{sum} = [draw, fill=white, circle, node distance=1cm]
 	\tikzstyle{input} = [coordinate]
 	\tikzstyle{output} = [coordinate]
 	\tikzstyle{pinstyle} = [pin edge={to-,thin,black}]
 	
 	
 	%Gauss distribution
 	\pgfmathdeclarefunction{gauss}{2}{%
 		\pgfmathparse{1/(#2*sqrt(2*pi))*exp(-((x-#1)^2)/(2*#2^2))}%
 	}
 % Bibliography style, important for biblatex functioning	
\bibliographystyle{apa}


	\maketitle
\begin{abstract}
The number of couples that are living together without being married represents a large share of total cohabiting couples in the United States as well as many other countries. Yet, very little is known about the reasons behind cohabitation. In this paper, we use data from the National Longitudinal Survey of Youth 1997 to shed light on the different mating behavior observed by education. Using a structural model of partnership choice where agents learn about the quality of their match, we find that for college graduates cohabitation is more of an investment good, used to gather information about the partner that they eventually marry, while for the others cohabitation is more of a consumption good, used as a cheap substitute of marriage.
\end{abstract}
\textbf{Keywords}: Marriage, Cohabitation, Divorce, Heterogeneous Agents, Match Quality Models,  Education, Structural Estimation\\
\textbf{JEL-Code}: D83 - J12 
\noindent \medskip{}
\end{document}